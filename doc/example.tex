\documentclass{article}

\usepackage{booktabs}
\usepackage{amsmath}
\usepackage{mathtools}
\usepackage{siunitx}
\usepackage{xcolor}

\begin{document}

Note that the way the contribution to the assignment of the variables associated with the other rows is not correct as described below.
The required formula is given in the paper.

We consider a problem with three equations:
\begin{align*}
x + y & \geq 2\\
2*x - y & \geq 0\\
-x + 2*y & \geq 1
\end{align*}

We obtain the following tableau with slack variables \(s_1\), \(s_2\), and \(s_3\) together with their bounds,
and the initial assignment assigning all variables to zero:
\begin{center}
\begin{tabular}{SSSSSc}
\toprule
\multicolumn{1}{c}{\(x\)} &
\multicolumn{1}{c}{\(y\)} &
\multicolumn{1}{c}{\(s_1\)} &
\multicolumn{1}{c}{\(s_2\)} &
\multicolumn{1}{c}{\(s_3\)} &
\multicolumn{1}{c}{bounds}\\
\midrule
\color{green}1  & \color{green}1  & -1 & 0  & 0  & \(\textcolor{red}{s_1 \geq 2}\) \\
2  & -1 & 0  & -1 & 0  & \(s_2 \geq 0\) \\
\color{green}-1 & \color{green}2  & 0  & 0  & -1 & \(\textcolor{red}{s_3 \geq 1}\) \\
\midrule
0  & 0  & 0  & 0  & 0\\
\bottomrule
\end{tabular}
\end{center}

The bounds of \(s_1\) and \(s_3\) are violated.
In both cases we can use \(x\) and \(y\) for pivoting because neither of them have bounds.
We pivot \(s_1\) and \(x\).
Observe that we can increase \(x\) by \((2 - 0)/1\) to set \(s_1\) to 2.
We begin with inverting the first row and eliminating \(x\) from the second and third row:
\begin{center}
\begin{tabular}{SSSSSc}
\toprule
\multicolumn{1}{c}{\(x\)} &
\multicolumn{1}{c}{\(y\)} &
\multicolumn{1}{c}{\(s_1\)} &
\multicolumn{1}{c}{\(s_2\)} &
\multicolumn{1}{c}{\(s_3\)} &
\multicolumn{1}{c}{bounds}\\
\midrule
-1 & -1  & 1 & 0  & 0  & \(s_1 \geq 2\) \\
0 & -3 & 2  & -1 & 0  & \(s_2 \geq 0\) \\
0 & 3  & -1  & 0  & -1 & \(s_3 \geq 1\) \\
\midrule
0  & 0  & 0  & 0  & 0\\
\bottomrule
\end{tabular}
\end{center}

Then, we swap the columns for \(x\) and \(s_1\), and update the assignment.
We increase the value of \(x\) by 2 and set the value of \(s_1\) to 2.
The remaining basic variables now have a contribution by \(s_1\) because of pivoting.
The contribution consists of the divisor of the first row, the value of \(s_1\), and the coefficient in the respective row.
We can simply add this contribution to the assignment.
\begin{center}
\begin{tabular}{cccccc}
\toprule
\multicolumn{1}{c}{\(s_1\)} &
\multicolumn{1}{c}{\(y\)} &
\multicolumn{1}{c}{\(x\)} &
\multicolumn{1}{c}{\(s_2\)} &
\multicolumn{1}{c}{\(s_3\)} &
\multicolumn{1}{c}{bounds}\\
\midrule
\(1\)        & \(-1\)             & \(-1\)  & \(0\)       & \(0\)          & \(s_1 \geq 2\) \\
\(2\)        & \(-3\)             & \(0\)   & \(-1\)      & \(0\)          & \(s_2 \geq 0\) \\
\(-1\)       & \(\color{green}3\) & \(0\)   & \(0\)       & \(-1\)         & \(\textcolor{red}{s_3 \geq 1}\) \\
\midrule
\(2\)        & \(0\)              & \(0+2\) & \(0+(1*2*2)\) & \(0+(1*2*(-1))\)\\
\(2\)        & \(0\)              & \(2\)   & \(4\)       & \(-2\)\\
\bottomrule
\end{tabular}
\end{center}

Now, only the lower bound of \(s_3\) is violated and only \(y\) is suitable for pivoting.
Observe that we can increase \(y\) by \((1-(-2))/3=1\) to set \(s_3\) to 1.
We divide the third row by \(-3\) and then eliminate \(y\) from the first and second row:
\begin{center}
\begin{tabular}{cccccc}
\toprule
\multicolumn{1}{c}{\(s_1\)} &
\multicolumn{1}{c}{\(y\)} &
\multicolumn{1}{c}{\(x\)} &
\multicolumn{1}{c}{\(s_2\)} &
\multicolumn{1}{c}{\(s_3\)} &
\multicolumn{1}{c}{bounds}\\
\midrule
\(2/3\) & \(0\)  & \(-1\) & \(0\)  & \(-1/3\) & \(s_1 \geq 2\) \\
\(1\)   & \(0\)  & \(0\)  & \(-1\) & \(-1\)   & \(s_2 \geq 0\) \\
\(1/3\) & \(-1\) & \(0\)  & \(0\)  & \(1/3\)  & \(s_3 \geq 1\) \\
\midrule
\(2\)   & \(0\)  & \(2\)  & \(4\)  & \(-2\)\\
\bottomrule
\end{tabular}
\end{center}

Then, we swap the columns for \(y\) and \(s_3\), and update the assignment.
We increase the value of \(y\) by 1 and set the value of \(s_3\) to 1.
The remaining basic variables now have a contribution by \(s_3\) because of pivoting.
We can simply add this contribution to the assignment.
\begin{center}
\begin{tabular}{cccccc}
\toprule
\multicolumn{1}{c}{\(s_1\)} &
\multicolumn{1}{c}{\(s_3\)} &
\multicolumn{1}{c}{\(x\)} &
\multicolumn{1}{c}{\(s_2\)} &
\multicolumn{1}{c}{\(y\)} &
\multicolumn{1}{c}{bounds}\\
\midrule
\(2/3\) & \(-1/3\) & \(-1\)             & \(0\)            & \(0\)     & \(s_1 \geq 2\) \\
\(1\)   & \(-1\)   & \(0\)              & \(-1\)           & \(0\)     & \(s_2 \geq 0\) \\
\(1/3\) & \(1/3\)  & \(0\)              & \(0\)            & \(-1\)    & \(s_3 \geq 1\) \\
\midrule
\(2\)   & \(1\)    & \(2+(3*1*(-1/3))\) & \(4+(3*1*(-1))\) & \(0+1\)\\
\(2\)   & \(1\)    & \(1\)              & \(4\)            & \(1\)\\
\bottomrule
\end{tabular}
\end{center}
\end{document}
